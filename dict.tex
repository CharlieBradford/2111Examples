\documentclass[a4paper, fleqn]{article}
\usepackage[margin=2cm]{geometry}
\usepackage{amsmath}
\usepackage{amssymb}
\usepackage{amsthm}
\usepackage{enumitem}
\usepackage{lastpage}
\usepackage{stmaryrd}
\usepackage[T1]{fontenc}
\usepackage{lmodern}
\usepackage{csquotes}
\usepackage{xfrac}
\usepackage{xcolor}
\usepackage{listings}
\usepackage{titlesec}
\usepackage[title]{appendix}
\usepackage{fancyhdr}
\usepackage[font=small, labelfont=bf]{caption}
\usepackage[english]{babel}
\usepackage{sectsty}
\usepackage{url, lipsum}
\usepackage[hidelinks]{hyperref}

\definecolor{mGreen}{rgb}{0,0.6,0}
\definecolor{mGray}{rgb}{0.5,0.5,0.5}
\definecolor{mPurple}{rgb}{0.58,0,0.82}
\definecolor{backgroundColour}{rgb}{0.95,0.95,0.92}

\lstdefinestyle{CStyle}{
    backgroundcolor=\color{backgroundColour},   
    commentstyle=\color{mGreen},
    keywordstyle=\color{magenta},
    numberstyle=\tiny\color{mGray},
    stringstyle=\color{mPurple},
    basicstyle=\footnotesize,
    breakatwhitespace=false,     
    breaklines=true,         
    captionpos=b,            
    keepspaces=true,         
    numbers=left,            
    numbersep=5pt,          
    showspaces=false,        
    showstringspaces=false,
    showtabs=false,          
    tabsize=2,
    language=C
}

\lstdefinestyle{customc}{
  belowcaptionskip=1\baselineskip,
  breaklines=true,
  frame=L,
  xleftmargin=\parindent,
  language=C,
  showstringspaces=false,
  basicstyle=\footnotesize\ttfamily,
  keywordstyle=\bfseries\color{green!40!black},
  commentstyle=\itshape\color{purple!40!black},
  identifierstyle=\color{blue},
  stringstyle=\color{orange},
}

\setlength{\mathindent}{20mm}

\newcommand{\D}{\texorpdfstring{\textsuperscript{D}}{D}}
\newcommand{\A}{\texorpdfstring{\textsuperscript{A}}{A}}
\newcommand{\reason}[1]{\text{\textcolor{blue}{$\langle$ #1 $\rangle$}}}
\newcommand{\num}[1]{\text{\textcolor{red}{(#1)}}}

\pagestyle{fancy}
\fancyhf{}
\setlength\headheight{15pt}
\fancyhead[L]{Trie Harder}
\fancyhead[R]{\rightmark}
\fancyfoot[R]{Page \thepage\ of \pageref{LastPage}}

\begin{document}

\title{ \normalsize \textsc{COMP2111}
        \\ [2.0cm]
		\rule{\linewidth}{0.5pt} \\
        \LARGE \textbf{\uppercase{Assignment 3} \\
		\normalsize Trie Harder}
		\rule{\linewidth}{2pt} \\ [0.5cm]
        \normalsize \today \vspace*{5\baselineskip}}

\date{}
\author{Charlie Bradford \\
		\texttt{z5114682}
		\and
		Julian Yu \\
		\texttt{z5135085}
		}


\vspace{2 in}
\maketitle

\tableofcontents


\clearpage\section{Syntactic Data Type: Dict}
We will define a dict to be a set, $W$, of all the words in the dict.

We take the syntactic data type Dict to be defined by the predicate 
$$\text{init\D}\ =\ (W\ =\ \langle\rangle)$$
and the following operations.

%\subsection{New Dict\D}
%\hspace*{10mm}\textbf{proc} \textit{newdict\D} () \\
%\hspace*{20mm}$[\textbf{TRUE},\ W = \langle\rangle]$

\subsection{Add Word\D}
\hspace*{10mm}\textbf{proc} \textit{addword\D} (\textbf{value} $x$: word) \\
\hspace*{20mm}$\quad x\ :\ [\textbf{TRUE},\ W = \langle x,\ W_0 \rangle]$ 

\subsection{Check Word\D}
\hspace*{10mm}\textbf{func} \textit{checkword\D} (\textbf{value} $x$: word): bool \\
\hspace*{20mm}$\quad \textbf{var}\ y\ \bullet\ x,\ y\ :\ [\textbf{TRUE},\ y\ =\ (x\ \in\ W)]; \textbf{return}\ y$

\subsection{Delete Word\D}
\hspace*{10mm}\textbf{proc} \textit{delword\D} (\textbf{value} $x$: word) \\
\hspace*{20mm}$\quad x\ :\ [x\ \in\ W,\ x\ \notin\ W]$


\clearpage\section{Refinement to DictA}

Before refinement several things must be stated. Our alphabet is $L$ and consists of the 26 letters of the Roman alphabet. All our words $w$ are in the set $L*$. We say a word $v$ is less than another word $w$ if $v$ is a prefix of $w$ ($\exists v'.vv'\ =\ w$).  \\

We define a Trie $t$ to be a mapping between words ($w\ \in\ L^*$) and boolean values ($t: L^*\ \mapsto\ \mathbb{B}$). $t$ has domain $Dom\ t\ \subseteq\ L^*$.There are several requirements of $t$:
\begin{description}
		\item[Minimum Trie] The smallest Trie domain possible is the empty word $\epsilon$
				\begin{itemize}
						\item See section \ref{sec:init}
				\end{itemize}
		\item[Prefix Condition] If a word is in the domain of the Trie, all prefixs of the word are as well
				\begin{itemize}
						\item $w\ \in\ Dom\ t\ \Rightarrow \forall\ v\ \in\ L^*.(\exists v'.(vv'\ =\ w)\ \Rightarrow\ (v\ \in\ Dom\ t))$ 
				\end{itemize}
\end{description}

We also define a relation $f$ between our abstract and concrete states. $f(t)\ =\ \{w\ |\ w\ \in\ Dom\ t.(t(w)\ =\ \textbf{TRUE})\}$.

We define our refinement predicate $r$ to equal $(a\ =\ f(t))$

\subsection{New Specifications}
\subsubsection{Initialisation Predicate}
								\label{sec:init}
\hspace*{20mm}$\quad\textit{init\D}\ =\ (t\ =\ \{\epsilon\ \mapsto\ \textbf{FALSE}\})$
\subsubsection{Add Word}
\hspace*{10mm}\textbf{proc} \textit{addword\D} (\textbf{value} $x$: word) \\
\hspace*{20mm}$\quad x,\ t:\ [\textbf{TRUE},\ x\ \in\ Dom\ t\ \land\ t(x)]$ 

\subsubsection{Check Word}
\hspace*{10mm}\textbf{func} \textit{checkword\D} (\textbf{value} $x$: word): bool \\
\hspace*{20mm}$\quad \textbf{var}\ y\ \bullet\ x,\ y\ :\ [\textbf{TRUE},\ y\ =\ (x\ \in\ Dom\ t_0\ \land\ t_0(x))]; \textbf{return}\ y$

\subsubsection{Delete Word}
\hspace*{10mm}\textbf{proc} \textit{delword\D} (\textbf{value} $x$: word) \\
\hspace*{20mm}$\quad x\ :\ [x\ \in\ Dom\ t\ \land\ t(x),\ \neg t(x)]$



\subsection{Proof Burdens}
\subsubsection{Initialisation}
$$\textit{init}^{C}\ \Rightarrow\ \exists a.(\textit{init}^{A}\ \land\ r(a,\ c)$$
\begin{align*}
		&\quad \textit{init}^{C}\\
		\Leftrightarrow&\quad \reason{definition of $\textit{init}^C$} \\
		&\quad t\ =\ \{\epsilon\ \mapsto\ \textbf{FALSE}\} \\
		\Rightarrow&\quad\reason{logic: create a set with existing information} \\
		&\quad \langle\rangle\ =\ \{x\ |\ x\in\ Dom\ t.(t(x)\ =\ \textbf{TRUE}\} \\
		\Rightarrow&\quad\reason{logic} \\
		&\quad\exists W.(W\ =\ \langle\rangle\ \land\ W\ =\ \{x\ |\ x\in\ Dom\ t.(t(x)\ =\ \textbf{TRUE})\})\\
		\Leftrightarrow&\quad \reason{definition of $\textit{init}^A$ and $r$} \\
		&\quad\exists\ a.(\textit{init}^A\ \land\ r(a,\ c))
\end{align*}

\subsubsection{Add Word}
$$\textit{pre}^A\ \land\ r(a,\ c)\ \Rightarrow\ \textit{pre}^{C}$$
\begin{align*}
		&\quad \reason{definition of $\textit{pre}^{A}$} \\
		&\quad \textbf{TRUE}\ \land\ r\\
		\Rightarrow&\quad \reason{definition of $\textit{pre}^{C}$} \\
		&\quad \textbf{TRUE} 
\end{align*}
This is trivial for any value of r.

$$\textit{pre}^A[\sfrac{a_0,\ x_0}{a,\ x}]\ \land r[\sfrac{a_0,\ x_0,\ c_0}{a,\ x,\ c}]\ \land \textit{post}^{C}\ \Rightarrow\ \exists a.\textit{post}^A\ \land\ r(a, c)$$
\begin{align*}
		&\quad \textit{pre}^A[\sfrac{a_0,\ x_0}{a,\ x}]\ \land r[\sfrac{a_0,\ x_0,\ c_0}{a,\ x,\ c}]\ \land \textit{post}^{C} \\
		\Leftrightarrow&\quad \reason{Defintion of $\textit{pre}^A$ and $\textit{post}^C$} \\
		&\quad \textbf{TRUE}[\sfrac{W_0}{W}]\ \land r(t,\ W)[\sfrac{t_0,\ W_0}{t,\ W}]\ \land\ \{x\ \in\ Dom\ t\ \land\ t(x)\}\\
		\Leftrightarrow&\quad \reason{logic: drop true conjunct and definition of $r$} \\
		&\quad \{W\ =\ \langle\ w\ |\ t(w)\rangle\}[\sfrac{W_0,\ t_0}{W,\ t}]\ \land\  x\ \in\ Dom\ t\ \land\ t(x) \\
		\Leftrightarrow&\quad \reason{perform subsitutions} \\
		&\quad W_0\ =\ \langle\ w\ |\ t_0(w)\rangle\ \land\ x\ \in\ Dom\ t\ \land\ t(x)\\
		\Rightarrow&\quad\reason{logic: extend quatification with $t(x)\ =\ \textbf{TRUE}$} \\
		&\quad W_0\ =\ \langle\ w\ |\ t_0(w)\rangle\ \land\ W\ =\ \langle W_0, x\rangle\ \land\ x\ \in\ Dom\ t\ \land\ t(x)\ \\
		\Rightarrow&\quad \reason{logic: last two conjucts equivalent to last conjuct in new statement} \\
		&\quad \exists a.(x\ \in\ a\ \land\ a\ =\ \langle\ w\ |\ t(w)\rangle) \\
		\Rightarrow&\quad \reason{Definition of $r$ and $\textit{post}^A$} \\
		&\quad \exists a.(\textit{post}^A\ \land\ r(a, c))
\end{align*}

\subsubsection{Check Word}
$$\textit{pre}^A\ \land\ r(a,\ c)\ \Rightarrow\ \textit{pre}^C$$
\begin{align*}
		&\quad \reason{definition of $\textit{pre}^A$} \\
		&\quad \textbf{TRUE}\ \land\ r\\
		\Rightarrow&\quad\reason{definition of $\textit{pre}^C$} \\
		&\quad \textbf{TRUE} 
\end{align*}
This is trivial for any value of r.

$$\textit{pre}^A[\sfrac{a_0,\ x_0}{a,\ x}]\ \land r[\sfrac{a_0,\ x_0,\ c_0}{a,\ x,\ c}]\ \land \textit{post}^{C}\ \Rightarrow\ \exists a.(\textit{post}^A\ \land\ r(a,\ c))$$
\begin{align*}
		&\quad\textit{pre}^A[\sfrac{a_0,\ x_0}{a,\ x}]\ \land r[\sfrac{a_0,\ x_0,\ c_0}{a,\ x,\ c}]\ \land \textit{post}^{C} \\
		\Leftrightarrow&\quad \reason{definition of $\textit{pre}^A$ and $\textit{post}^C$} \\
		&\quad \textbf{TRUE}[\sfrac{W_0}{W}]\ \land\ r[\sfrac{W_0,\ t_0}{W,\ t}]\ \land\ \{y\ =\ (x\ \in\ Dom\ t_0\ \land\ t_0(x))\}\\
		\Leftrightarrow&\quad\reason{logic: drop true conjunct and definition of $r$} \\ 
		&\quad\{W\ =\ \langle\ x\ |\ t(x)\rangle\}[\sfrac{W_0,\ t_0}{W,\ t}]\ \land\ \{y\ =\ (x\ \in\ Dom\ t_0\ \land t_0(x))\} &&\\
		\Leftrightarrow&\quad\reason{perform substitutions} \\
		&\quad W_0\ =\ \langle\ x\ |\ t_0(x)\rangle\ \land\ y\ =\ (x\ \in\ Dom\ t_0\ \land t_0(x)) \\
		\Rightarrow&\quad\reason{logic: extend quatification with $t(x)\ =\ \textbf{TRUE}$} \\
		&\quad W_0\ =\ \langle\ w\ |\ t_0(w)\rangle\ \land\ W\ =\ \langle W_0, x\rangle\ \land\ y\ =\ (x\ \in\ Dom\ t_0\ \land\ t_0(x))\\
		\Rightarrow&\quad\reason{logic} \\
		&\quad\exists\ a.(y\ =\ (x\ \in\ a)\ \land\ a\ =\ \langle\ w\ |\ t(w)\rangle) \\
		\Leftrightarrow&\quad\reason{definition of $r$ and $\textit{post}^A$}\\
		&\quad\exists\ a.(\textit{post}^A \land\ r(a,\ c))
\end{align*}

\subsubsection{Delete Word}
 $$\textit{pre}^A\ \land\ r(a,\ c)\ \Rightarrow\ \textit{pre}^C$$
 \begin{align*}
		 &\quad\textit{pre}^A\ \land\ r(a,\ c) \\
		 \Leftrightarrow&\quad \reason{definition of $\textit{pre}^A$ and $r(a,\ c)$} \\
		 &\quad x\ \in\ W\ \land\ W\ =\ \langle\ w\ |\ t(w)\ \rangle \\
		 \Rightarrow&\quad\reason{logic} \\
		 &\quad t(x)\ =\ \textbf{TRUE} \\
		 \Rightarrow&\quad\reason{logic} \\
		 &\quad x\ \in\ Dom\ t\ \land\ t(x) \\
		 \Rightarrow&\quad\reason{definition of $\textit{pre}^C$} \\
		 &\quad \textit{pre}^C
 \end{align*}

 $$ \textit{pre}^A[\sfrac{a_0,\ x_0, \ W_0}{a,\ x, W}]\ \land r[\sfrac{a_0,\ x_0, \ W_0}{a,\ x}]\ \land \textit{post}^{C}\ \Rightarrow\ \exists a.\textit{post}^A\ \land\ r$$

\begin{align*}
		&\quad\textit{pre}^A[\sfrac{a_0,\ x_0, \ W_0}{a,\ x, W}]\ \land r[\sfrac{a_0,\ x_0, \ W_0}{a,\ x}]\ \land \textit{post}^{C}\\
		\Leftrightarrow&\quad\reason{definition of $\textit{pre}^A,\ r,$ and $\textit{post}^C$} \\
		&\quad\{x\ \in\ W\}[\sfrac{W_0,\ x_0}{W,\ x}]\ \land\ r[\sfrac{t_0,\ W_0}{t,\ W}]\ \land\ \{\neg t(x)\}\\
		\Leftrightarrow&\quad\reason{definition of r and perform substitutions} \\
		&\quad x_0\ \in\ W_0\ \land\ W_0\ =\ \{w\ |\ t_0(w)\}\ \land\ \neg t(x) \\
		\Rightarrow&\quad\reason{logic} \\
		&\quad x_0\ \in \ w_0\ \land\ w_0\ =\ \{w\ |\ t_0(w)\}\ \land\ x\ \notin\ \{w\ |\ t(w)\} \\
		\Rightarrow&\quad\reason{logic: for every concrete state there is an abstract state (see 2.2.5)} \\
		&\quad x_0\ \in \ w_0\ \land\ w_0\ =\ \{w\ |\ t_0(w)\}\ \land\ \exists W.(x\ \notin\ W\ \land\ W = \{w\ |\ t(w)\}) \\
		\Rightarrow&\quad\reason{logic :$(\phi\ \Rightarrow\ \psi)\ \Leftrightarrow (\phi\ \land\ \chi\ \Rightarrow\ \psi)$} \\
		&\quad \exists W.(x\ \notin\ W\ \land\ W\ =\ \{w\ |\ t(w)\}) \\  
		\Leftrightarrow&\quad\reason{definition of $\textit{post}^A$ and $r$}
\end{align*}
 

\subsubsection{Falsifiable Precondition}
$$\forall\ c.(\exists a.\ r)$$
\begin{align*}
		&\quad\reason{definition of $c,\ a,\ \text{and},\ r$} \\
		&\quad\forall\ t.(\exists\ W.\ W\ =\ \{w\ |\ t(w)\} \\ 
\end{align*}

This is trivial as the values in the domain of t that map to true is just some subset of $L*$, and we can pick any $W$ that is the same subset.


\clearpage\section{Derivation to Toy Language}
\subsection{New Dict}
\textbf{proc} \textit{newdict} (\textbf{value} $t$)
\begin{align*}
		&\quad t:\ [\textbf{TRUE},\ t\ =\ \{\epsilon\ \mapsto\ \textbf{FALSE}\}]\quad\num{1} \\
		\num{1}\sqsubseteq&\quad\reason{ass} \\
		&\quad t\ :=\ \{\epsilon\ \mapsto\ \textbf{FALSE}\};
\end{align*}

\begin{align*}
		&\textbf{proc}\ \textit{newdict}\ (\textbf{value}\ t) \\
		&\qquad t\ :=\ \{\epsilon\ \mapsto\ \textbf{FALSE}\};
\end{align*}



\subsection{Add Word}
\textbf{proc} \textit{addword} ($\textbf{value}\ x,\ \textbf{value}\ y,\ t$)

This starts with the non-trivial precondition $x\ \in\ Dom\ t$, which could be false for all initial values of $x$ other than $\epsilon$, however, if we require that $x\ =\ \epsilon$ whenever the proc is called externally, then then precondition must always be true. Thus, as the operation is only called in the state space where $\epsilon\ \in\ Dom\ t$ and is only called with $y\ =\ \epsilon$, the precondition is non-falsifiable and equivalent to the true precondition given earlier. For the mathemematically rigourous conditions of the state space see the introduction to section 2.
\begin{align*}
		&\quad x,\ y,\ t:\ [x\ \in\ Dom\ t,\ x.y\ \in\ Dom\ t\ \land\ t(x.y)]\quad\num{1} \\
		\num{1} \sqsubseteq&\quad\reason{if} \\
		&\quad\textbf{if } y = \text{`\textbackslash0'} \textbf{ then} \\
		&\qquad x,\ y,\ t:\ [x\ \in\ Dom\ t\,\ x.y\ \in\ Dom\ t\ \land\ t(x.y)]\quad\num{2} \\
		&\quad\textbf{else} \\
		&\qquad x,\ y,\ t:\ [x\ \in\ Dom\ t,\ x.y \ \in\ Dom\ t\ \land\ t(x.y)]\quad\num{3} \\ 
		&\quad\textbf{fi;} \\
		\num{2} \sqsubseteq&\quad\reason{func-ass} \\
		&\quad t\ :=\ (t:\ x\ \mapsto\ \textbf{TRUE}); \\
		\num{3}\sqsubseteq&\quad\reason{seq} \\
		&\quad x,\ y,\ t:\ [x\ \in\ Dom\ t,\ x.y[0]\ \in\ Dom\ t]\quad\num{4} \\
		&\quad x,\ y,\ t:\ [x.y[0]\ \in\ Dom\ t,\ x.y\ \in\ Dom\ t\ \land\ t(x.y)] \num {5} \\
		\num{4}\sqsubseteq&\quad\reason{if} \\
		&\quad\textbf{if } x.y[0]\ \notin\ Dom\ t\textbf{ then} \\
		&\qquad x,\ y,\ t:\ [x.y[0]\ \notin\ Dom\ t, x.y[0]\ \in\ Dom\ t]\quad\num{6} \\
		&\quad\textbf{else} \\
		&\qquad x,\ y,\ t:\ [x.y[0]\ \in\ Dom\ t, x.y[0] \in Dom\ t]\quad\num{7} \\
		&\quad\textbf{fi;} \\
		\num{5} \sqsubseteq&\quad\reason{procedure call} \\
		&\quad \textit{addword}(x.y[0], y[1..], t); \\
		\num{6} \sqsubseteq&\quad\reason{ass} \\
		&\quad Dom\ t\ := Dom\ t\ \cup\ \{x.y[0]\}; \\
		\num{7} \sqsubseteq&\quad\reason{skip} \\
		&\quad \textbf{skip};
\end{align*}

\begin{align*}
		&\textbf{proc}\ \textit{addword}\ (\textbf{value}\ x,\ \textbf{value}\ y,\ t) \\
		&\qquad \textbf{if } y\ =\ \text{`\textbackslash0'}\textbf{ then} \\
		&\qquad\qquad t\ :=\ (t:\ x\ \mapsto\ \textbf{TRUE}); \\
		&\qquad \textbf{else} \\
		&\qquad\qquad\textbf{if } x.y[0]\ \notin\ Dom\ t\textbf{ then} \\
		&\qquad\qquad\qquad Dom\ t\ :=\ Dom\ t\ \cup\ \{x.y[0]\}; \\
		&\qquad\qquad\textbf{else} \\
		&\qquad\qquad\qquad \textbf{skip};\\
		&\qquad\qquad\textbf{fi;} \\
		&\qquad\qquad\textit{addword}(x.y[0],\ y[1...],\ t); \\
		&\qquad \textbf{fi;}
\end{align*}



\subsection{Check Word}
\textbf{func} \textit{checkword} ($\textbf{value}\ \textit{prefix},\ \textbf{value}\ x,\ t,\ \textbf{return}\ y$)
\begin{align*}
		&\quad \textbf{var}\ y\ \bullet\ \textit{prefix},\ x,\ y,\ t\ :\ [\textbf{TRUE},\ y\ =\ (x\ \in\ Dom\ t\ \land\ t(x))]; \textbf{return}\ y\quad\num{1} \\
		\num{1}\sqsubseteq&\quad\reason{if} \\
		&\quad \textbf{if }x\ =\ \text{`\textbackslash0'}\textbf{ then} \\
		&\qquad \textit{prefix},\ x,\ y,\ t\ :\ [\textit{prefix}.x\ \in\ Dom\ t,\ y\ =\ (\textit{prefix}.x\ \in\ Dom\ t\ \land\ t(\textit{prefix}.x))]\quad\num{2}\\
		&\quad\textbf{else} \\
		&\qquad \textit{prefix},\ x,\ y,\ t\ :\ [\textit{prefix}\ \in Dom\ t,\ y\ =\ (\textit{prefix}.x\ \in\ Dom\ t\ \land\ t(x))] \quad\num{3} \\
		&\quad\textbf{fi;} \\
		\num{2}\sqsubseteq&\quad\reason{logic and ass} \\
		&\quad y\ :=\ t(x); \\
		\num{3}\sqsubseteq&\quad\reason{if} \\
		&\quad\textbf{if} \textit{prefix}.x[0]\ \in\ Dom\ t\textbf{ then} \\
		&\qquad \textit{prefix},\ x,\ y,\ t\ :\ [\textit{prefix}.x[0]\ \in\ Dom\ t,\ y\ =\ (\textit{prefix}.x\ \in\ Dom\ t\ \land\ t(\textit{prefix}.x))]\quad\num{4} \\
		&\quad\textbf{else} \\
		&\qquad \textit{prefix},\ x,\ y,\ t\ :\ [\textit{prefix}.x[0]\ \notin\ Dom\ t,\ y\ =\ (\textit{prefix}.x\ \in\ Dom\ t\ \land\ t(\textit{prefix}.x))]\quad\num{5} \\
		&\quad\textbf{fi;} \\
		\num{4}\sqsubseteq&\quad\reason{function call} \\
		&\quad y\ :=\ \textit{checkword}\ (\textit{prefix}.x[0],\ x[1..],\ t,\ y) \\
		\num{5}\sqsubseteq&\quad\reason{logic (\textbf{prefix condition})} \\
		&\quad y\ :=\ \textbf{FALSE}
\end{align*}

\begin{align*}
		&\textbf{func}\ \textit{checkword}(\textbf{value}\ \textit{prefix},\ \textbf{value}\ x,\ t,\ \textbf{return}\ y) \\ 
		&\qquad\textbf{if } y[0]\ =\ \text{`\textbackslash0'}\textbf{ then} \\
		&\qquad\qquad b\ :=\ t(x); \\
		&\qquad\textbf{else} \\
		&\qquad\qquad\textbf{if } x.y[0]\ \in\ Dom\ t\textbf{then} \\
		&\qquad\qquad\qquad b\ :=\ \textit{checkword}(x.y[0],\ w[1..],\ t);\\
		&\qquad\qquad\textbf{else} \\
		&\qquad\qquad\qquad b\ :=\ \textbf{FALSE};\\
		&\qquad\qquad\textbf{fi;}\\
		&\qquad\textbf{fi;}\\
		&\qquad \textit{return}\ b;
\end{align*}

\subsection{Delete Word}
\textbf{proc} \textit{delword} ($\textbf{value}\ x,\ \textbf{value}\ y,\ t$)
\begin{align*}
		&\quad x,\ y,\ t:\ [xy\ \in\ Dom\ t\ \land\ t(x),\ \neg t(x)]\quad\num{1} \\
		\num{1}\sqsubseteq&\quad\reason{if} \\
		&\quad \textbf{if } y\ =\ \text{`\textbackslash0'} \textbf{ then} \\
		&\qquad x,\ y,\ t:\ [xy\ \in\ Dom\ t\ \land\ t(xy),\ \neg t(x)];\quad\num{2} \\
		&\quad \textbf{else} \\
		&\qquad x,\ y,\ t:\ [x\ \in\ Dom\ t\ \land\ t(xy),\ \neg t(x)]; \quad\num{3} \\
		&\quad \textbf{fi;} \\
		\num{2}\sqsubseteq&\quad\reason{ass} \\
		&\quad t\ :=\ (t:\ x\mapsto\ \textbf{FALSE}) \\
		\num{3}\sqsubseteq&\quad\reason{procedure call} \\
		&\quad \textit{delword}(x.y[0],\ y[1..],\ t);
\end{align*}

\begin{align*}
		&\textbf{proc}\ \textit{delword}\ (\textbf{value}\ x,\ \textbf{value}\ y,\ t) \\
		&\qquad\textbf{if } y\ =\ \text{`\textbackslash0'} \textbf{ then} \\
		&\qquad\qquad t\ :=\ (t:\ x\ \mapsto\ \textbf{FALSE}); \\
		&\qquad\textbf{else} \\
		&\qquad\qquad \textit{delword}(x.y[0],\ y[1..],\ t); \\
		&\qquad\textbf{fi;}
\end{align*}


\clearpage\section{Translation to C}
\subsection{Code}
\begin{lstlisting}[style=customc]
void newdict (Dict *dp)
{
    (*dp) = malloc(sizeof(Dict));
    (*dp)->eow = FALSE;
    int i;
    for (i = 0; i < VECSIZE; i++)
	{
        (*dp)->cvec[i] = calloc(0, sizeof(Dict));
		(*dp)->cvec[i] = NULL;
    }
}


void addword (const Dict r, const word w)
{
    if (w[0] == 0)
        r->eow = TRUE;
    else
    {
        if (r->cvec[w[0] - 'a'] == NULL)
            newdict(&(r->cvec[w[0] - 'a']);
		else
		    \\ SKIP
        addword(r->cvec[w[0] - 'a'], w + sizeof(char));
    }
}


void checkword (const Dict r, const word w)
{
    bool b;
    if (w[0] == 0)
        b = r->eow;
    else
	{
        if (r->cvec[w[0] - 'a'] == NULL)
            b = FALSE;
		else
            b = checkword(r->cvec[w[0] - 'a'], w + sizeof(char));
    }
    return b;
}


void delword (const Dict r, const word w)
{
    if (w[0] == 0)
        r->eow = FALSE;
    else
        delword(r[w[0] - 'a'], w + sizeof(char));
}
\end{lstlisting}

\subsection{Changes}
\begin{itemize}
		\item In the toy language we could access $t(w)$ for any $w$ at any point, however in C we can only check one letter at a time, and the rest of the word is made up of context (nodes already traversed)
		\item In the toy language we simply pass $t$ to the recursive function, whereas in C we must pass the appropriate trie node
		\item In the toy language it is enough to set $t\ :=\ (\epsilon\ \mapsto\ \textbf{FALSE})$ however in C we need to allocate memory and set the values of all the sub-nodes to NULL
		\item When we add a word or prefix to $Dom\ t$ in the toy language we just take the union of $t$ with the mapping, however in c we must add a new node to the Trie and treat it like a fresh Trie. 
\end{itemize}




\end{document}
